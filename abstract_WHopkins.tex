\documentclass[letter, USenglish, 11pt, subfigure]{article}
\usepackage[margin=1in]{geometry}
\newcommand*{\ATLASLATEXPATH}{./}
\usepackage{\ATLASLATEXPATH atlaspackage}
\usepackage{\ATLASLATEXPATH atlasbiblatex}
\usepackage{\ATLASLATEXPATH atlasphysics}
\usepackage{\ATLASLATEXPATH atlasjetetmiss}
\usepackage{\ATLASLATEXPATH ANA-SUSY-2018-12-PAPER-defs}
\usepackage{enumerate}
\newcommand{\mm}{\ensuremath{\mu^{+}\mu^{-}}}
\newcommand{\bsmm}{\ensuremath{\Bs\to\mm}}
\newcommand{\bdmm}{\ensuremath{\Bd\to\mm}}
\usepackage{lineno}
%\linenumbers
\usepackage{wrapfig}
\usepackage{placeins}

\pagestyle{headings}

\title{Early Career Proposal: \\Machine-Learning-Driven New Physics Searches at the Large Hadron Collider}
\author{Walter Hopkins, Argonne National Laboratory}
\date{}
\begin{document}
\pagenumbering{gobble}

\maketitle
\abstract{
The Standard Model (SM) of particle physics has been highly successful. However, it does not have explanations for several phenomena, including dark matter and the large differences in the strengths of fundamental forces. After discovering the last piece of the SM, the Higgs boson, experiments at the Large Hadron Collider (LHC) have been searching for hints of physics Beyond the SM (BSM) to yield insights into these phenomena. These searches have not yet produced any significant deviations from SM predictions.
The LHC will be upgraded to provide a significant increase in luminosity (HL-LHC), resulting in a tenfold increase in the data set. This increase brings both new opportunities for BSM physics discovery as well as computational challenges, such as the vastness of the data set that needs to be probed for hints of new physics and the need to simulate large amounts of SM backgrounds. Machine learning (ML) and the upcoming exascale High Performance Computing (HPC) resources will provide promising tools to tackle these HL-LHC computing challenges. The proposed research focuses on developing an automated BSM search strategy to probe the far reaches of the HL-LHC data set. Furthermore, to estimate the SM background in these regions, the research will aim to improve the computational speed of ATLAS detector simulations with ML. The search strategy will be generated by ML clustering algorithms that will group theory models with similar experimental signatures together. The computational speed of ATLAS simulations will be improved by tuning simulation configurations while ML algorithms will ensure minimal loss of accuracy. By systematically constructing search regions and accelerating simulations, this approach will enable ATLAS to maximally exploit the HL-LHC data set for BSM physics discovery.
}

\end{document}